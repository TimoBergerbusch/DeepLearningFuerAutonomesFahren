\chapter{Appendix}
\vspace*{-4em}
\begin{figure}[H]
\begin{minipage}[t]{0.33\textwidth}
	\lstinputlisting[numbers=left, basicstyle=\scriptsize,firstline=1, lastline=51]{src/listing/alexnet.prototxt}
\end{minipage}
\begin{minipage}[t]{0.33\textwidth}
	\lstinputlisting[numbers=left, firstnumber = 51, basicstyle=\scriptsize,firstline=51, lastline=101]{src/listing/alexnet.prototxt}
\end{minipage}
\begin{minipage}[t]{0.33\textwidth}
	\lstinputlisting[numbers=left, firstnumber = 101, basicstyle=\scriptsize,firstline=101, lastline=139]{src/listing/alexnet.prototxt}
	\hspace*{2cm}\vdots
\end{minipage}

\caption{The \caffe implementation of the \alexnet. This is only the first 169 lines. The whole net has a size of 284 lines of code. The main reason of this is the highly verbose Protocol Buffer prototxt style. Further explanation in \Cref{sec: Caffe}. \cite{CaffeAlexNetGithub}}
\label{lst: Caffe AlexNet}
\end{figure}

\begin{figure}
	%	\centering
	\todoLine
	\hspace*{-0.5cm}
	\lstinputlisting[numbers = left, basicstyle=\scriptsize]{src/listing/alexnet.txt}
	\todoLine
	\caption{The \alexnet implemented in \cnnarch. This is the whole program to describe the whole net. Further explanation in \Cref{sec: CNNArch}. \cite{CNNArch}}
	\label{lst: CNNArchLang AlexNet}
\end{figure}