{\bf\Large Abstract} \\ [1em]
The topic of autonomous driving using artificial intelligence increases in importance with the overwhelming amount of software usage within vehicles. For that \textit{Convolutional Neural Networks} (CNNs), which try to figure out the importance of special areas of a single picture, have been shown to be promising. \\
In this paper we will give a general introduction to the topic of CNNs.
We distinguish between the three main \textit{deep learning languages} (DLLs) currently used and researched for autonomous driving agents: mediated perception, behaviour reflex and direct perception.\\
Further we will compare different languages, which can be used to implement the different DLLs, based on the factors of usability, scope of functionality and the integration on a subject.\\
As a proof of concept we will train a CNN using the language \textit{CNNArch} on the famous KITTI dataset in order to create a trained model. This model will then be tested on a test set created using either the simulation tool \texttt{MontiSim} or the open source racing game TORCS, containing multiple different challenging scenarios the agent has to manage.\\
Finally we evaluate the trained model on it's performance and try to reason, why it performed particularly good/bad, and give an overview based on the implemented test in order to state the similarities and differences of the languages.

\cleardoublepage
