{\bf\Large Abstract} \\ [1em]
The topic of autonomous driving using artificial intelligence increases in importance with the overwhelming amount of software usage within vehicles. For that \textit{Convolutional Neural Networks} (CNNs), which try to figure out the importance of special areas of a single picture, have been shown to be promising. \\
In this paper we will give a general introduction to the topic of \nns and then state the specialties defining a CNN.
We distinguish between the three main paradigms currently used and researched for autonomous driving agents: mediated perception, behavior reflex and direct perception.\\
Further we will compare different domain specific languages \cnnarch, \caffe, \caffetwo and \mxnet, based on the factors of usability, scope of functionality, also regarding training possibilities, and the integration on a subject.

We will analyze a CNN using the direct perception approach, in different languages, and compare it to state-of-the-art implementations of the other paradigms, training on a simulator \torcs or the \kitti database. Further we state scenarios probably causing problems for the direct perception approach. 

Finally we create an overview over the mentioned languages with a table, which states the functionalities and properties in a nutshell.
\cleardoublepage
